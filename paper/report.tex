\documentclass[12pt]{article}
\usepackage[utf8]{inputenc}
\usepackage[english]{babel}
\usepackage{booktabs}
\usepackage{hyperref}
\usepackage{geometry}
\geometry{a4paper, margin=2.5cm}

\title{Assignment 1: Basic Matrix Multiplication in Different Languages}
\author{Jaime Rivero Santana \\ Degree in Data Science and Engineering \\ University of Las Palmas de Gran Canaria}
\date{October 20, 2024}

\begin{document}

\maketitle

\begin{abstract}
This report presents a comparative study of the performance of the basic matrix multiplication algorithm ($O(n^3)$) implemented in three programming languages: C, Java, and Python. Execution times were measured for square matrices of increasing sizes (100, 200, 500, and 1000). Results show that C and Java achieve highly efficient performance, while Python is significantly slower due to its interpreted nature.
\end{abstract}

\section{Results}
\begin{center}
\begin{tabular}{crrr}
\toprule
Matrix size ($n$) & C (s) & Java (s) & Python (s) \\
\midrule
100 & 0.001843 & 0.001331 & 0.051941 \\
200 & 0.008965 & 0.006810 & 0.404254 \\
500 & 0.119080 & 0.082889 & 6.560263 \\
1000 & 0.956151 & 0.766837 & 55.027762 \\
\bottomrule
\end{tabular}
\end{center}

\section{Conclusion}
Language choice significantly impacts performance. C and Java are efficient for compute-intensive tasks, while Python requires optimized libraries (e.g., NumPy) for such workloads.

\section*{Repository}
\url{https://github.com/JaimeRiveroSantana/Task1-Language-Benchmark-of-matrix-multiplication}

\end{document}
